% -*-latex-*-
%
% For questions, comments, concerns or complaints:
% thesis@mit.edu
%
%
% $Log: cover.tex,v $
% Revision 1.8  2008/05/13 15:02:15  jdreed
% Degree month is June, not May.  Added note about prevdegrees.
% Arthur Smith's title updated
%
% Revision 1.7  2001/02/08 18:53:16  boojum
% changed some \newpages to \cleardoublepages
%
% Revision 1.6  1999/10/21 14:49:31  boojum
% changed comment referring to documentstyle
%
% Revision 1.5  1999/10/21 14:39:04  boojum
% *** empty log message ***
%
% Revision 1.4  1997/04/18  17:54:10  othomas
% added page numbers on abstract and cover, and made 1 abstract
% page the default rather than 2.  (anne hunter tells me this
% is the new institute standard.)
%
% Revision 1.4  1997/04/18  17:54:10  othomas
% added page numbers on abstract and cover, and made 1 abstract
% page the default rather than 2.  (anne hunter tells me this
% is the new institute standard.)
%
% Revision 1.3  93/05/17  17:06:29  starflt
% Added acknowledgements section (suggested by tompalka)
%
% Revision 1.2  92/04/22  13:13:13  epeisach
% Fixes for 1991 course 6 requirements
% Phrase "and to grant others the right to do so" has been added to
% permission clause
% Second copy of abstract is not counted as separate pages so numbering works
% out
%
% Revision 1.1  92/04/22  13:08:20  epeisach

% NOTE:
% These templates make an effort to conform to the MIT Thesis specifications,
% however the specifications can change.  We recommend that you verify the
% layout of your title page with your thesis advisor and/or the MIT
% Libraries before printing your final copy.
\title{The Design and Implementation of a Congestion Control Plane}

\author{Akshay Narayan}
%\prevdegrees{S.B., Massachusetts Institute of Technology (2017)}
% If you wish to list your previous degrees on the cover page, use the
% previous degrees command:
%       \prevdegrees{A.A., Harvard University (1985)}
% You can use the \\ command to list multiple previous degrees
%       \prevdegrees{B.S., University of California (1978) \\
%                    S.M., Massachusetts Institute of Technology (1981)}
\department{Department of Electrical Engineering and Computer Science}

% If the thesis is for two degrees simultaneously, list them both
% separated by \and like this:
% \degree{Doctor of Philosophy \and Master of Science}
\degree{Master of Engineering in Electrical Engineering and Computer Science}

% As of the 2007-08 academic year, valid degree months are September,
% February, or June.  The default is June.
\degreemonth{February}
\degreeyear{2019}
\thesisdate{November 1, 2018}

%% By default, the thesis will be copyrighted to MIT.  If you need to copyright
%% the thesis to yourself, just specify the `vi' documentclass option.  If for
%% some reason you want to exactly specify the copyright notice text, you can
%% use the \copyrightnoticetext command.
%\copyrightnoticetext{\copyright IBM, 1990.  Do not open till Xmas.}

% If there is more than one supervisor, use the \supervisor command
% once for each.
\supervisor{Hari Balakrishnan}{Fujitsu Chair Professor}

% This is the department committee chairman, not the thesis committee
% chairman.  You should replace this with your Department's Committee
% Chairman.
\chairman{Leslie A. Kolodziejski}{Chair, Department Committee on Graduate Theses}

% Make the titlepage based on the above information.  If you need
% something special and can't use the standard form, you can specify
% the exact text of the titlepage yourself.  Put it in a titlepage
% environment and leave blank lines where you want vertical space.
% The spaces will be adjusted to fill the entire page.  The dotted
% lines for the signatures are made with the \signature command.
\maketitle

% The abstractpage environment sets up everything on the page except
% the text itself.  The title and other header material are put at the
% top of the page, and the supervisors are listed at the bottom.  A
% new page is begun both before and after.  Of course, an abstract may
% be more than one page itself.  If you need more control over the
% format of the page, you can use the abstract environment, which puts
% the word "Abstract" at the beginning and single spaces its text.

%% You can either \input (*not* \include) your abstract file, or you can put
%% the text of the abstract directly between the \begin{abstractpage} and
%% \end{abstractpage} commands.

% First copy: start a new page, and save the page number.
\cleardoublepage
% Uncomment the next line if you do NOT want a page number on your
% abstract and acknowledgments pages.
% \pagestyle{empty}
\setcounter{savepage}{\thepage}
\begin{abstractpage}
This paper describes the implementation and evaluation of a system to implement complex congestion control functions by placing them in a separate agent outside the datapath.  
Each datapath---such as the Linux kernel TCP, UDP-based QUIC, or kernel-bypass transports like mTCP-on-DPDK---summarizes information about packet round-trip times, receptions, losses, and ECN via a well-defined interface to algorithms running in the off-datapath Congestion Control Plane (CCP). 
The algorithms use this information to control the datapath's congestion window or pacing rate. Algorithms written in CCP can run on multiple datapaths. CCP improves both the pace of development and ease of maintenance of congestion control algorithms by providing better, modular abstractions, and supports aggregation capabilities of the Congestion Manager, all with one-time changes to datapaths. 
CCP also enables new capabilities, such as Copa in Linux TCP, several algorithms running on QUIC and mTCP/DPDK, and the use of signal processing algorithms to detect whether cross-traffic is ACK-clocked.
Experiments with our user-level Linux CCP implementation show that CCP algorithms behave similarly to kernel algorithms, and incur modest CPU overhead of a few percent.

\end{abstractpage}

% Additional copy: start a new page, and reset the page number.  This way,
% the second copy of the abstract is not counted as separate pages.
% Uncomment the next 6 lines if you need two copies of the abstract
% page.
% \setcounter{page}{\thesavepage}
% \begin{abstractpage}
% This paper describes the implementation and evaluation of a system to implement complex congestion control functions by placing them in a separate agent outside the datapath.  
Each datapath---such as the Linux kernel TCP, UDP-based QUIC, or kernel-bypass transports like mTCP-on-DPDK---summarizes information about packet round-trip times, receptions, losses, and ECN via a well-defined interface to algorithms running in the off-datapath Congestion Control Plane (CCP). 
The algorithms use this information to control the datapath's congestion window or pacing rate. Algorithms written in CCP can run on multiple datapaths. CCP improves both the pace of development and ease of maintenance of congestion control algorithms by providing better, modular abstractions, and supports aggregation capabilities of the Congestion Manager, all with one-time changes to datapaths. 
CCP also enables new capabilities, such as Copa in Linux TCP, several algorithms running on QUIC and mTCP/DPDK, and the use of signal processing algorithms to detect whether cross-traffic is ACK-clocked.
Experiments with our user-level Linux CCP implementation show that CCP algorithms behave similarly to kernel algorithms, and incur modest CPU overhead of a few percent.

% \end{abstractpage}

\cleardoublepage

\section*{Acknowledgments}
I first thank my current adult supervision: Hari Balakrishnan and Mohammad Alizadeh, as well as my former advisors, Sylvia Ratnasamy and Scott Shenker.

Second, I thank my CCP colleagues for their ideas, code, feedback, friendship, and support: Frank Cangialosi, Deepti Raghavan, Prateesh Goyal, Srinivas Narayana, and Radhika Mittal.

Third, I thank all the other members of the Networks and Mobile Systems group at MIT CSAIL who have made my time thus far here so enjoyable, especially Anirudh Sivaraman, Ravi Netravali, Amy Ousterhout, Vikram Nathan, Vibhaa Sivaraman, and Venkat Arun.

Fourth, I thank the members of the UC Berkeley Netsys Lab who first taught me how fun research can be, especially Rachit Agarwal, Justine Sherry, Aurojit Panda, Peter Gao, and Gautam Kumar.

Fifth, and lastly, I thank my friends, especially Sagar Karandikar, Shoumik Palkar, and Paroma Varma, and most importantly my family: Amma, Appa, Madhuri, and Rohan, for all their support.

% -*-latex-*-
