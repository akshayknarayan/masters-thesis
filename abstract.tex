% $Log: abstract.tex,v $
% Revision 1.1  93/05/14  14:56:25  starflt
% Initial revision
%
% Revision 1.1  90/05/04  10:41:01  lwvanels
% Initial revision
%
%
%% The text of your abstract and nothing else (other than comments) goes here.
%% It will be single-spaced and the rest of the text that is supposed to go on
%% the abstract page will be generated by the abstractpage environment.  This
%% file should be \input (not \include 'd) from cover.tex.
% In this thesis, I designed and implemented a compiler which performs
% optimizations that reduce the number of low-level floating point operations
% necessary for a specific task; this involves the optimization of chains of
% floating point operations as well as the implementation of a ``fixed'' point
% data type that allows some floating point operations to simulated with integer
% arithmetic.  The source language of the compiler is a subset of C, and the
% destination language is assembly language for a micro-floating point CPU.  An
% instruction-level simulator of the CPU was written to allow testing of the
% code.  A series of test pieces of codes was compiled, both with and without
% optimization, to determine how effective these optimizations were.
In this thesis, I extend the pluggable congestion control interface in QUIC (QUIC UDP Internet Connections) by integrating support for the recently proposed Congestion Control Plane architecture into QUIC.
The Congestion Control Plane (CCP) removes congestion control logic off the datapath into a separate agent.
Now, algorithm developers can write their algorithm into the userspace CCP API and automatically run their algorithm on different CCP enabled datapaths, such as QUIC and the Linux Kernel.
This thesis explores integrating support for a CCP datapath into QUIC (Quic UDP Internet Connections), so congestion control algorithms written in CCP can automatically run in QUIC.
In addition to modifying the congestion control interface in QUIC itself. this includes designing a general CCP datapath shared library, libccp, Libccp handles the common responsibilities common to the kernel and QUIC datapath.
The evaluation focuses on three aspects: (1) Do algorithms written through CCP have the same general performance as QUIC algorithms?
(2) Does the CCP API allow developers to take advantage of everything the current QUIC pluggable congestion control interface offers?
Finally, I compare performance of algorithms, not written in QUIC before, across QUIC and the Linux Kernel.
